\documentclass[]{article}
%This is what NCWIT asks.
%opening
\title{A Family Centered Introduction to Computer Science, with Analogies between Cooking Recipes  and Programming, with a Theme of Contribution to Disaster Recovery}
\author{}

\begin{document}

\maketitle

 

 

\section{Principle investigator’s name(s) and email(s)}
Sixia Chen, schen@ccsu.edu\\
Th\'er\`ese Smith, therese.smith@ccsu.edu
\section{Institution and department name}
Central Connecticut State University, department of Computer Science
\section{Proposal summary (40 words)}
We propose a family dinner, with demonstrations of procedural steps of cooking interleaved with programming (We have analogies for iteration, recursion, trees and graphs.) followed by a robotics contest, and a book signing, for our food and code cookbook.


\section{Proposal overview (500 words)}
We propose to prototype an annual outreach in the Hartford area.
We propose to reach out to families with girls in middle or high school, and present to them a heartwarming learning experience involving computing. By offering a meal in a setting with the family, we wish to obtain family appreciation of a student's interest in CS.
We use the idea of connecting CS with something students can be passionate about by having a speaker on CS and a topic of the year, starting with disaster relief. We are geographically close to Trinity College, where humanitarian free and open software is a research topic; there is software related to disaster relief in Prof. Heidi Ellis's portfolio. We have invited Professor Ellis. Our CCSU President Toro has been working on helping students from Puerto Rico. We have invited these students to attend, to lend immediacy to the talk about disaster relief.  We are geographically close to UCONN/Storrs. They have expertise in discrete optimization, including applied to disaster relief, with a riveting animation of evacuation route planning. We would like to invite someone to speak about that.
We are also situated among health care and insurance industry, and hope in future years to draw on these for other areas of passion to be connected to computer science.
The meal will be accompanied by the cooking demonstrations that illustrate the programming concepts, with a lecture about the analogous computer science principles. 
A cookbook with recipes and computer programs and references to programming is being prepared. We plan to have a book signing at the event.
There are robotics clubs, both college and high school, with which we would like to choreograph an entertainment. Perhaps we could imagine a robotic assisted search-and-rescue operation when we have the disaster relief theme. %284 words
We plan to give out T-shirts with the text "Computer Science puts bread on the table.", with an image of a bread basked in which the bread is shaped like dollar signs.%329 words


\section{Alignment with NCWIT-promoted practices (whether Program-in-a-Box or Promising
Practices) found here (300 words): ncwit.org/higheredresources}
In alignment with promising practice \#8, we showcase career opportunities by having a speaker who is a job recruiter or a member of our industrial liason group. Several of these people are women, thus will also serve as role models. The robots are controlled through Scratch, so we will show the students how they can drag-and-drop instructions to make a program, and to move a robot, through Scratch. 
By using a family dinner, we plan to ``prime the pump'' on family based promising practices, which is to say, we incorporate the family's presence, and we offer a cookbook that includes recipes for food and also recipes for computer programs in Scratch and Java, including directions for obtaining these. 
We plan to display the book "Engineering the ABS's: How Engineers Shape our World.
We plan to have computers logged in to  ``Stories of 8 Female Black Inventors", and to ``Program Your Own Robot'', and to ``Latinas in Computing''. We plan to ask our visitor from Puerto Rico to help with the Spanish language website. There are also many Spanish speaking families in Willimantic, CT, a somewhat nearby city. We plan to include the ``Watch the Big Dream'' video at the doorway of our dinner, and run it repeatedly (in a loop) so that people waiting to get in will be able to watch it.

We will include flyers for junior membership in the Society of Women Engineers. We will include lists of free computer science courses, both in the ``Cooking With Java, Python'' book ant at the literature table.

For growth mindset, Chef Jay will emphasize that he learned his cooking techniques. Professor Chen will point out that she learned her programming techniques.

The event will showcase several role models, including Professor Chen and Dr. Smith.

We will include spatial intelligence by featuring at one table how our department head teaches software engineering with Lego.%314 words


\section{Implementation plan and schedule (300 words)}
Prepare the recipes with analogies to computer science cookbook.
Obtain the literature to display.
Make the photo for the Tshirts, and obtain the Tshirts.
Plan out the demonstrations.
Plan out the entertainment.
Arrange with the source of the dinner.
Prepare the invitations.

\section{Budget justification—maximum \$10,000 (300 words)}
We determine how many people we can serve by subtracting single costs, such as the book production, Tshirts, invitations, the projectors for the cooking and computer science demonstrationsthe room (and literature and computer tables) rental from the total grant amound, and dividing by the per place meal cost.
\section{Evaluation plan (300 words)}
We will survey the students as they leave about their enthusiasm for computer science careers with a Likert scale.
We will offer a ``free text'' commentary area.
We will consult with the recruitment and admissions professionals about their opinion, both in the planning and in the evaluation stage.
We will perform a qualitative analysis on the comments, using the method of Braun and Clarke,(Successful Qualitative ResearchA Practical Guide for Beginners
Virginia Braun - University of Auckland, New Zealand, Victoria Clarke - University of The West of England, UK, 2013, SAGE Publishing) to find what main themes occurred in the commentary. This will allow us to plan future events with more knowledge about what about the theme and activities had positive and negative impact.%123 words
\section{Sustainability plan (300 words)}
If the recruitment office believes this is a successful approach, they might choose to fund it in the future.
\section{Letter of recommendation from chair or dean (submitted as a separate PDF)}

\end{document}
